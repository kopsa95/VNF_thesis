Το κεφάλαιο συνήθως συγκεντρώνει και σχολιάζει
τα αποτελέσματα της εργασίας, είτε πρόκειται για θεωρητική εργασία
είτε για πειραματική.

Είναι το σημαντικότερο κεφάλαιο στην εργασία, όπου γίνεται σχολιασμός
και κριτική της εργασίας και των αποτελεσμάτων της και στο οποίο μπορεί
να περι\-λαμβάνονται πίνακες, όπως οι Πίνακες~\ref{tab:no1} και ~\ref{tab:no2}

\begin{table}[ht]
\centering
\begin{tabular}{c|cc}
R. Feynman & 1 & 2 \\
P. Higgs & 3 & 4 \\
L. Boltzmann & 5 & 6
\end{tabular}
\caption{Ένα παράδειγμα 3$\times$3 πίνακα με κεντρική στοίχιση και μία κάθετη διαγράμμιση}
\label{tab:no1}
\end{table}

\begin{table}
    \begin{tabular}{lll}
    1 & f & d           \\
    2 & g & sfh         \\
    3 & h & jfgjdjfgjfg \\
    4 & j & jkfdjfgjdf  \\
    \end{tabular}
\end{table}

\begin{table}[ht]
\centering
\begin{tabular}{l|cc}
R. Feynman & 1 & 2 \\\hline
P. Higgs & 3 & 4 \\\hline
L. Boltzmann & 5 & 6
\end{tabular}
\caption{Ένα παράδειγμα 3$\times$3 πίνακα με αριστερή στοίχιση και οριζόντιων και κάθετων διαγραμμίσεων}
\label{tab:no2}
\end{table}

Για τις επιλογές και ρυθμίσεις των πινάκων, οι οποίες προσφέρουν ιδιαίτερα
πολλές δυνατότητες, παραπέμπουμε το συγγραφέα στην αναφορά \cite{wiki_latex}.
Για τις εικόνες ή σχήματα που περιλαμβάνονται
στην εργασία ισχύουν ανάλογα, όπως το παρακάτω παράδειγμα (Εικ. \ref{fig:uoa_logo}).

\begin{figure}[ht]
\centering
\includegraphics[width=0.3\textwidth]{uoa_new_logo_red.png}
\caption{Η κεφαλή της Αθηνάς - Επίσημο λογότυπο του ΕΚΠΑ}
\label{fig:uoa_logo}
\end{figure}