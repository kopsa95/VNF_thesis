There is a rising demand for services consisting of multiple
interconnected components, e.g., microservices in a service
mesh or chained virtual network functions (VNFs) in network function virtualization (NFV) \cite{12}. These services can
scale flexibly by instantiating service components according
to current demand. Such instances can run independently on
any compute node in the network and process incoming flows
requesting the service.
The goal of service coordination is to ensure that these
flows are processed successfully by traversing instances of
all required service components. Additionally, flows should
complete with short end-to-end delay to ensure good Quality
of Service (QoS). To this end, the requested services need to be
scaled and their instances placed in the network, i.e., we have
to decide how many service components to instantiate where.
Furthermore, incoming flows need to be routed from their
ingress nodes through these deployed instances and finally to
their egress nodes. In doing so, node and link capacities need
to be respected and link delays should be considered.

To address the issue three algorithms proposed were evaluated, a simulation environment was deployed proposed at \cite{12}
Algorithm used: Random Schedule,Load Balance algorithm and Shortest Path algorithm (more details of these algorithms will be discussed in the next section)

Real graph network topologies were used as input with "fake" traffic produced from the author, in order to simulate the mentioned algorithms.