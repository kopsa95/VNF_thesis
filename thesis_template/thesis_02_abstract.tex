% print no page number
\thispagestyle{empty}


% The abstract in english
\begin{center}
\large
{\bf Abstract}\\[5mm]
\end{center}
Network Function Virtualization (NFV) is the current concept used from the majority of the operators that decouples
network functions (such as firewalls, DNS, NATs, load balancers, etc.) from dedicated
hardware devices (the traditional expensive middleboxes). This decoupling enables hosting
of network services, known as Virtualized Network Functions (VNFs), on commodity hardware (such as switches or servers) and thus facilitates and accelerates service deployment
and management by providers, improves flexibility, leads to efficient and scalable resource
usage, and reduces costs. This paradigm is a major turning point in the evolution of networking, as it introduces high expectations for enhanced economical network services, as
well as major technical challenges.

