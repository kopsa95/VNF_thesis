Nowadays, Information Technology companies and academics use more and
more about the Network Function Virtualization (NFV) concept. In fact, Communications
Service Providers (CSPs) and Internet Service Providers (ISPs) are facing competition from
Over-the-top (OTT) media services and web services, experiencing declining average revenue per user and feeling the pressure to innovate rapidly to respond to new trends
such as 5G(preparing 6G as well), IoT (Internet of Things), and cloud edge computing.

Traditional network services are built by chaining together proprietary single-function boxes
(middleboxes). The design of these services is custom, the thechnology used is expensive, require lengthy prior analysis before implementing and most of the times cannot be shared with any other service.
Once deployed, the operations and management of these services is largely non-automated,
with each box presenting its own management interface. This technique of creating network
services is very expensive, and offers no practical way of creating dynamic services.

Network Function Virtualization (NFV) is the trend-technology currently used by most of the operators, that can greatly assist in solving
these business challenges. Once virtualized, the Virtual Network Functions (VNFs) can
be hosted on an industry-standard server or commodity hardware. Virtualization does not
stop at replacing physical boxes with virtual machines, but can go further by using microservices, containers, and cloud native architectures. Managing the Lifecycle of these
services (such as initial deployment, configuration changes, upgrades, scale-out, scale-in,
self-healing, etc.), can also be automated. These VNFs can also be chained and managed in a dynamic and automated fashion. All these advances enable the creation and management
of agile network services.

This thesis addresses some baseline algorithms along with its results,
 of these virtualized network functions and services in a simulation-environment. That is the problem for coordination of service mesh consisting of
  multiple microservices. Includes Non-RL algorithms (Random Schedule,
Shortest Path, and Load Balance). This topic is always under
constant analysis and research  
from many operators , as the coordination of the services is a complicated
a problem and proposals for better solutions are currently
analyzed from many Research Departments.