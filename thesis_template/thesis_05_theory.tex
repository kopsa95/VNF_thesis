Το κεφάλαιο της βιβλιογραφικής ανασκόπησης στοχεύει στη συγκέντρωση
της υπάρχουσας βιβλιογραφίας σχετικής με το αντικείμενο της εργασίας.
Η βιβλιογραφική ανασκόπηση έχει ως κύριο στόχο να παράσχει το υπάρχον
επιστημονικό υπόβαθρο που απαιτείται για τη θεμελίωση της επιστημονικής
εγκυρότητας της εργασίας.

Για εργασίες πειραματικού χαρακτήρα, το κεφάλαιο αυτό μπορεί εναλλακτικά
να περιλάβει το θεωρητικό μέρος. Παρακάτω παρατίθενται τρία παραδείγματα
εξισώσεων: ένα εντός της παραγράφου: $\sum_i x_i = K$, ένα σε ξεχωριστή γραμμή,
αλλά χωρίς αρίθμηση:
\[
\sum_i x_i=K
\]
κι ένα πλήρες, με αρίθμηση στα δεξιά και εισαγωγή ετικέτας ώστε να γίνεται
αναφορά προς αυτήν:
\begin{equation}
\label{eq:example}
	\sum_i x_i = K
\end{equation}
Η αναφορά γίνεται με παρόμοιο τρόπο με τις εικόνες και τους πίνακες.

